
\section{Conclusion}

This research presented **Causal Discovery Kolmogorov-Arnold Networks (CD-KAN)**, a novel information systems artifact that bridges the gap between predictive accuracy and causal interpretability in time series analysis. By integrating the expressive power of Kolmogorov-Arnold Networks with rigorous differentiable DAG learning, we demonstrated that it is possible to achieve state-of-the-art forecasting performance without sacrificing the ability to understand the underlying causal mechanisms.

Our comprehensive evaluation across ten diverse datasets confirmed CD-KAN's superiority over existing methods, achieving a mean F1-score of 0.8971 in causal discovery (a 37\% improvement over the best baseline) and reducing forecasting error by an order of magnitude in financial applications. More importantly, we showed how CD-KAN serves as a "dual-capability" decision support system, empowering managers with both accurate foresight and actionable insight.

Future research will extend this work in three directions: (1) Scaling the DAG constraint optimization to high-dimensional systems ($d > 100$) via approximate algorithms; (2) Incorporating domain knowledge constraints to guide structure learning in data-scarce environments; and (3) Developing interactive visualizations to allow managers to perform "what-if" analyses based on the discovered causal graphs.

As organizations increasingly rely on AI for strategic decision-making, the need for systems that are both accurate and trustworthy is paramount. CD-KAN represents a significant step towards this goal, offering a principled path for integrating deep learning into the managerial toolkit.
